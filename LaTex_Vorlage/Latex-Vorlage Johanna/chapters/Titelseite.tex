% Deckblatt des Dokuments 

%% das hier einkommentieren, wenn nicht das RWTH-Logo genutzt wird
%\thispagestyle{empty}
%\begin{titlepage}
%\vspace*{-2cm}
%\begin{figure*}
%\vspace*{-1.7cm}
%\raisebox{-3,5mm}
%{\includegraphics[width=1.0\textwidth]{./Bilder/avt}}
%\end{figure*}

%% das hier einkommentieren, wenn das Corporate Design verwendet wird

%Einf�gen des Institutskopf
\thispagestyle{empty}
\begin{titlepage}
\vspace*{-3cm}
\hfill 
%\raisebox{-7,5mm}
{\includegraphics[width=12cm]{./grafiken/rwth_ltt_en_rgb.png}} % hier nach Bedarf das deutsche Logo einbinden
%\vspace*{-0.5cm}

%\begin{picture}(0,0)
%\raisebox{3mm}{\line(0,0){0}}
%\end{picture}

\begin{minipage}[t]{16cm}
\begin{flushleft}
%\normalsize{\textbf{Diese Arbeit wurde vorgelegt am Lehrstuhl f�r Systemverfahrenstechnik} \newline 
%The present work was submitted to Chair of Process Systems Engineering}\\ %Institut eintragen!
\normalsize{\textbf{The present work was submitted to Chair of Technical Thermodynamics}}\\
\end{flushleft}
\end{minipage}
\vspace{2.5cm}
% Ende Instituskopf

\begin{center}
\begin{LARGE}
Enter a fancy title\\ % Titel eintragen
\end{LARGE}
\vspace{1cm}
\large{Bachelor-/ Masterthesis} % Art der Arbeit eintragen
\end{center}
% Beginn Titelseite
\begin{flushleft}

\vspace{2cm}
%\normalsize{vorgelegt von}\\
%\normalsize{\textbf{Nachname, Vorname}}\\
%\normalsize{Matrikelnummer xxx xxx}\\
%\normalsize{Bachelor-, Master-, Diplom-, Magisterarbeit}\\
%\vspace{2.5cm}
\normalsize{presented by}\\ % nach Bedarf die deutsche Version einkommentieren
\normalsize{\textbf{Nachname, Vorname}}\\ % Name eintragen
\normalsize{StudentID no. xxx xxx}\\ % Matrikelnummer eintragen
\vspace{2cm}
\normalsize{\textbf{Supervisor:}}\\	%richtigen Erstpr�fer eintragen
\normalsize{Vorname Nachname, M.Sc.}\\		%richtigen Zweitpr�fer eintragen

\vspace{2cm}
\normalsize{$1^{st}$ Examiner: Univ.-Prof. Dr.-Ing. Andr� Bardow}\\	%richtigen Erstpr�fer eintragen (evt ins Deutsche �bersetzen)
\normalsize{$2^{nd}$ Examiner: Jun.-Prof. Dr. Kai Leonhard}\\		%richtigen Zweitpr�fer eintragen
\end{flushleft}
\vfill

\begin{flushright}
\normalsize{Aachen, September 30, 2016} %VORSICHT: \today-Befehl gibt aktuelles Datum (z.B. "10. Juli 2015", bei Einstellung auf englisch "July 10, 2015") aus ggf. in Abgabedatum �ndern!
\end{flushright}

\end{titlepage}

\thispagestyle{empty}
\cleardoublepage