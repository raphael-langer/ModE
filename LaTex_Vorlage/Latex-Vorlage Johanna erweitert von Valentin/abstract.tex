%In the Paris Agreement, the United Nations agreed to approach anthropogenic climate change, in order to mitigate its adverse consequences. Therefore, the emissions of greenhouse gases have to be limited. As the  chemical industry is a large scale source of greenhouse gas emissions, existing and new production processes for chemicals have to be analyzed from an environmental perspective. To do so
To evaluate chemical products from an environmental perspective, \acf{LCA} represents the current standard. % to perform environmental assessments of production processes. 
LCA requires large amounts of data that can be sourced from LCA databases. However, LCA databases are limited in the number of listed chemicals. In order to increase the availability of datasets, simplified process design methods can be used to calculate mass and energy flows of chemical processes. These can be used as a basis to built LCA databases. Simplified process design methods deliver mass and energy flows of chemical processes on the basis of empirical equations, thermodynamic considerations and assumptions, being less data intense than collecting manufacture data or performing process simulations. %However, the effects of applying \acl{spdm}s to calculate inventory data have not been assessed in the context of existing database methodologies yet. 
%This thesis describes the \aclp{spdm} stoichiometry, the Ecoinvent default value approach, thermodynamic considerations of the reaction, distillation, pumps and compressors and empirical fugitive emission factors. 
A qualitative analysis of \aclp{spdm} to calculate mass flows, heat and electricity demands as well as fugitive emissions reveals limitations in the scope of the methods. The scope ignores the influence of ancillary material and additional product treatment steps, such as purifying. Additionally, the ability of \aclp{spdm} to deliver appropriate results depends on appropriate parameters, such as equipment efficiencies.

In the case study, a method based on stoichiometric equations and the heat of reaction as well as a method based on stoichiometric equations and default energy values are implemented to generate mass and energy flows of chemical processes and to generate a LCA database. The LCA databases generated by both methods are compared to a database which is based on industrial data. For the two methods the root-mean-square error of all global warming impacts included in the database are 1.20 kg CO$_2$-eq and 1.08 kg CO$_2$-eq for the reaction enthalpy based method and the default values based method, respectively. Energy requirements are underestimated by far, while the default value of 2 MJ/kg heat use performs better than using reaction enthalpies for heat demands, because almost all processes considered need heat, even if they are exothermic. Unconsidered heat requirements result from preheating, separation and purification process steps. Energy demands and material inputs are identified to be the major driver of the global warming impact of chemical products. Under- or overestimation of these values result in high errors in the LCA database. Mass flows calculated on the basis of stoichiometry match the original values well with a root-mean-square error of 0.42 kg. More than 50 \% of the results lie within an error range of 25 \%. Thus, using the introduced \aclp{spdm} for the generation of LCA databases is appropriate for mass flows. However, energy flows should not be approximated by the used \aclp{spdm}.