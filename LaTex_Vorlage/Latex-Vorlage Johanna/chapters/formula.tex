\makenomenclature %Nomenklatur wird erstellt, aber noch nicht ausgegeben


\renewcommand{\nomname}{List of Abbreviations and Symbols Used} % Deutsche �berschrift evt einf�gen
\newcommand{\einheit}[1]{\renewcommand{\nomentryend}{\hfill $\left[ #1 \right]$} } % Einheiten k�nnen angegeben werden 

% ++++ Damit Nomenklatur im Anhang als eigenes Chapter aufgef�hrt wird 
\setlength{\nomlabelwidth}{20mm}  % LabelBreite
\makeatletter   
\renewcommand{\thenomenclature}{%
\chapter{\nomname}
\list{}{
\labelwidth\nom@tempdim
\leftmargin\labelwidth
\advance\leftmargin\labelsep
\let\makelabel\nomlabel
}
}
\makeatother
% ++++ Ende eigenes Chapter

% ++++ Die einzelnen Teile der Nomenklatur werden definiert ++++
\renewcommand{\nomgroup}[1]{%
	\ifthenelse{\equal{#1}{S}}{\item[\textbf{List of Symbols}]}{%
		\ifthenelse{\equal{#1}{A}}{\item[\textbf{List of Abbreviations}]} {
		  \ifthenelse{\equal{#1}{I}}{\item[\textbf{List of indices}]} {} } }	
		 }

% +++++++  Ab hier die Nomenklatureintr�ge bearbeiten ++++++++++++

% VERZEICHNIS DER FORMELZEICHEN
\nomenclature[S]{$\varphi$}{smoothed switching function between 0 and $\Delta h_i$ \einheit{-}}


% VERZEICHNIS DER ABK�RZUNGEN
 \nomenclature[A]{$DAE$}{Differential-algebraische Gleichung}%


% VERZEICHNIS DER INDIZES
\nomenclature[I]{$UBD$}{upper bounding}%


