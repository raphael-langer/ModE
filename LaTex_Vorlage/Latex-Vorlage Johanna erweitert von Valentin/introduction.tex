Anthropogenic greenhouse gas emissions are the largest driver of climate change. Even small increases in global average temperature (e.g. + 0.5 °C)  will have primarily negative consequences on human health  \cite{HoeghGuldberg.2018}. In the Paris Agreement  \cite{UnitedNations.2015}, 195 nations agreed to limit the global warming to 2 °C in order to mitigate the negative effects of climate change. This requires emitting less greenhouse gases to the atmosphere; particularly carbon dioxide (CO$_2$)  \cite{RogeljJ.ShindellD..2018}. 
Among all industries, the chemical industry consumes the most energy: 10 \% of the total worldwide final energy demand. Based on the extensive use of fossil fuels, the chemical industry contributes to 7\%  of the global greenhouse gas emissions \cite{InternationalEnergyAgency.2013}. Because of the relevance of the chemical industry regarding greenhouse gas emissions, environmentally preferable processes have to be identified. Thus, chemical products have to be evaluated from an environmental perspective. To evaluate chemical products from an environmental perspective, \acf{LCA} is widely used. The interference of a product system with the environment is described in the mass and energy inventory, which lists all flows exchanged with the environment  \cite{InternationalOrganizationforStandardization.2006}. %Based on the inventory, the environmental impacts of the product system can be calculated  \cite{InternationalOrganizationforStandardization.2006}.

However, compiling the inventory needs a vast amount of data about mass and energy flows entering or leaving a product system  \cite{InternationalOrganizationforStandardization.2006}. To facilitate data collection, \ac{LCA} practitioners search for inventory  and impact data in \ac{LCA} databases. Yet, only about 500 out of 85000 commercially produced chemicals are listed in LCA databases \cite{Parvatker.2019}. Furthermore, the quality of inventory data differs. This is because \ac{LCA} database providers use a variety of approaches for generating inventory data  \cite{Hischier.2005, Parvatker.2019}: chemical plant data is considered to be the most precise, even though errors can occur from measurement  \cite{Parvatker.2019}. Process simulations generate precise inventories, as long as assumptions and equipment data is correct. However, process simulation requires much time and expert knowledge \cite{Parvatker.2019}. Process calculations use empirical design equations, thermodynamic considerations and information from literature; for example the stoichiometric equation. Because process calculations need less resources than process simulations, process calculations are widely used in LCA databases \cite{Althaus.2007}. Yet, the results may suffer from wrong assumptions and a limited scope  \cite{Parvatker.2019}. Statistical approaches like neuronal networks are a top-down approach, those results depend on quality and quantity of training data. Besides that, neuronal networks and their results are ``black boxes'' \cite{Parvatker.2019}. Proxy chemicals substitute inventories of other chemicals that are not inventoried. Consequentially, only chemicals that are inventoried can serve as proxy. However, finding appropriate proxies requires deep knowledge of similar production paths \cite{Parvatker.2019}. Default values approaches -- like the Ecoinvent approach -- substitute a process specific inventory by average values; introducing an uncertainty into the inventory result \cite{Hischier.2005}. 

Among these methods, this thesis focuses on basic process calculation and the Ecoinvent default value approach, which are called ``simplified process design methods'' here. Simplified process design methods are most commonly used for the generation of mass and energy inventories \cite{Parvatker.2019,Hischier.2005,Althaus.2007} and have a low data requirement. Thus, \aclp{spdm} are a promising approach for the fast and automated generation of inventory data, out of which \ac{LCA} databases are generated. However, the effects of applying \acl{spdm}s to calculate mass and energy inventories have not been assessed yet in the context of LCA database generation.

To fill this gap, chapter \ref{chap:LCA} presents how methodological choices in LCA affect LCA database results. Chapter \ref{spdm} identifies and summarizes the most important \aclp{spdm} as well as qualitatively analyses how  LCA results are influenced by the use of \aclp{spdm}. In a case study in Chapter \ref{chap:case study}, two \aclp{spdm} are applied to generate a database of ecological impacts of chemical products. In the course of this,  an existing databases framework, developed at the Chair of Technical Thermodynamics at RWTH Aachen University is used. Finally, in Chapter \ref{chap:results}, the new \ac{LCA} database results are compared to a \ac{LCA} database, generated with industrial data. By these means, the applicability of \aclp{spdm} is assessed with regard to consequences when generating \ac{LCA} databases.