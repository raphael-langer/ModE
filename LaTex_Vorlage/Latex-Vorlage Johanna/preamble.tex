\documentclass[
12pt,						% Schriftgröße
DIV10,
german,						% für Umlaute etc
listof=totoc,				% Verzeichnisse im Inhaltsverzeichnis aufführen
bibliography=totoc, 		% Literaturverzeichnis im Inhaltsverzeichnis aufführen
index=totoc,				% Index im Inhaltsverzeichnis aufführen
oneside,
paper=a4,
headings=normal,
titlepage,					% es wird eine Titelseite verwendet
parskip=half,				% Abstand zwischen Absätzen (halbe Zeile)	
headsepline,        		% Linie unter Kopfzeile
final
]{scrbook} % alternativ: scrreprt, scrartcl


% Packages
% -----------------------------------
%Einstellungen der Seitenränder
\usepackage[inner=3.5cm,outer=2cm,top=2.5cm,bottom=2cm,includeheadfoot]{geometry}

%% Zitate
\usepackage{apacite} 					% Literatur-Referenzen: American Psycholog. Assoc.
\usepackage{natbib}						% definiert den Zitierstil als Autor (Jahr)
\usepackage{mdwlist}
\setcitestyle{round,aysep={}} 			% Indizierg. in runden Klammern, zw. Autor u. Jahr


%% Überschriften in Deutsch
% \usepackage[latin1]{inputenc} 		% Umlaute im Text
\usepackage[ansinew]{inputenc}        	% Umlaute ermöglichen - einkommentieren, falls oben nicht 															funktioniert
\usepackage[ngerman, english]{babel}   	% englische und deutsche Sprache, die letzte Sprache gibt an, in 													welcher Sprache die Überschriften (Literaturverzeichnis vs. 													References etc) angegeben werden
\usepackage[T1]{fontenc}
\hyphenation{voll-st\"andigen}

%%% Bilder
\usepackage{graphicx} 					% Grafiken einfügen (pdf,png - aber jpg vermeiden)
\graphicspath{{./Bilder/}}              % Pfad zu den Bildern
\usepackage{pdflscape}               	% Einzelne Seiten im Querformat möglich
\usepackage[pdftex]{xcolor}             % Eigene Farbdefinition möglich
\usepackage{subfigure}               	% für mehrere Grafinken in einer Figure
\usepackage{placeins}                	% Bessere Positionierung von Tabellen und Figures [\FloatBarrier]
\usepackage{float}						% Stellt die Platzierungsoption[H] zur Verfügung, mit der man flaots 												an die Stelle zwingen kann, wo sie definiert wurden.

%% für schöne Graphiken (können so direkt in Latex erstellt werden, sehr praktisch!)
\usepackage{tikz}
\usetikzlibrary{shapes.geometric}
\usetikzlibrary{shapes.arrows}
\usepackage{pgfplots}					% ermöglicht das Erstellen von Plots in der Tikzpicture-Umgebung
\usepackage{filecontents}				% ermöglicht das Einlesen der zu plottenden Daten aus .csv oder .txt-											  Datein

%%% Tabellen
\usepackage{longtable}               	% größere Tabellen mit Umbrüchen möglich 
\usepackage{ifthen}						% ermöglicht if-Abfragen
\usepackage{rotating}
\usepackage{array} 						% Für schönere und vielfältigere Tabellen
\usepackage{multirow} 					% Mehrere Spalten/Zeilen in einer Tabelle zusammenfassen 															(multicolumn, multirow}
\usepackage{booktabs}					% Das Paket stellt zwei Design-Paradigmen auf: 1. Vertikale Linien 													sind böse,  2. Doppelte Linien sind böse

% Ein paar Neudefinitionen zum Anpassen des multirow-Befehls an die booktabs-Umgebung
		\newcommand{\forloop}[5][1]{%
		\setcounter{#2}{#3}%
		\ifthenelse{#4}{#5\addtocounter{#2}{#1}%
		\forloop[#1]{#2}{\value{#2}}{#4}{#5}}%
		{}}
		\newcounter{crcounter}
		\newcommand{\compensaterule}[1]{%
		\forloop{crcounter}{1}{\value{crcounter} < #1}%
		{\vspace*{-\aboverulesep}\vspace*{-\belowrulesep}}}
		\newcommand{\multirowbt}[3]{\multirow{#1}{#2}%
		{\compensaterule{#1}#3}}
		
		\usepackage{epstopdf}
\usepackage{url}					  
\usepackage{tabularx} 				% bessere Gestaltung von Tabellen
\usepackage[active]{srcltx}
\usepackage{setspace} 				% Zeileneinstellung: 1,5-facher Zeilenabstand

%% Formeln
\usepackage{amsmath} 				% bessere Formelumgebung
\usepackage{amsfonts}               % Mehr Schriftarten in Formeln
\usepackage{amssymb}                % Sonderzeichen in Formeln
\usepackage{siunitx}				% ermöglicht Darstellung der Einheiten in SI Einheiten
\usepackage{textcomp}				% ermöglicht °C und € im Fließtext (mit \textcelsius) 

%% Verzeichnisse
\usepackage{nomencl}                % fürs Symbolverzeichnis	
\usepackage{booktabs}

%TODO
% Load the package
\usepackage{glossaries}
% Generate the glossary
\makeglossaries
%TODO end
\usepackage[ruled,vlined]{algorithm2e}
\newtheorem{mydef}{Merksatz}  		% Falls Beispiele, Merksätze m. fortl. Nr. gebr. werden
\newtheorem{bsp}{Beispiel}
\rmfamily 							% Serifenschrift
% \usepackage[margin=0.5in]{geometry}
\addtolength{\oddsidemargin}{.15in}
\addtolength{\evensidemargin}{-.15in}

\renewcommand{\familydefault}{\sfdefault}	
%\renewcommand{\familydefault}{\rmdefault}	

%% für Code im Anhang
\usepackage{listings} \lstset{numbers=left, numberstyle=\tiny, numbersep=5pt}
\lstset{language=Matlab,breaklines=true}
\usepackage{color}

\parindent0pt %Erstzeileneinzug abschalten


\newcounter{mytemp}
\newcounter{BlockCounter}
 \renewcommand{\theBlockCounter}{\Alph{BlockCounter}} 		% nummeriert Blöcke in Flowcharts mit A,B,C statt 1,2,3
 
 \addtocounter{secnumdepth}{1}
\addtocounter{tocdepth}{1}

% Alle Querverweise und URLs als Link darstellen
% In der PDF-Ausgabe
\usepackage[colorlinks=true, bookmarks, bookmarksnumbered, bookmarksopen, bookmarksopenlevel=1,
linkcolor=black, citecolor=black, urlcolor=black, filecolor=black,
pdfpagelayout=OneColumn, pdfnewwindow=true,
pdfstartview=Fit, plainpages=false, pdfpagelabels, pdftex,
pdfauthor={Vorname Nachname},
pdftitle={Titel der Arbeit},
pdfsubject={Masterarbeit},
pdfkeywords={Keyword1, Keyword2, Keyword3, ...}]{hyperref}

% Farben und Logos siehe: https://intranet.avt.rwth-aachen.de/Intranet/AVT/wiki/index.php/Logos#Farben  
	% Farbennamen und -definitionen sind an die AVT-Powerpointpräsentation angelehnt. 
%	\definecolor{avtcolor}{rgb}{0,0.537,0.820}	% Mit "rgb" muss man von den Angaben aus Word (= denen von der Webseite) immer Umrechnen
%	\definecolor{avtcolor}{RGB}{14,133,196}			% mit "RGB" (großgeschrieben) muss man nicht Umrechen!
%	\definecolor{rwthblue}{RGB}{0,83,159}
%	\definecolor{186blau}{rgb}{0.729 ,0.859, 0.941} % ganz helles blau
%	\definecolor{135blau}{rgb}{0.529,0.82,0.941} % auch ganu helles Blau
%	\definecolor{64blau}{rgb}{0.25,0.741,0.827} % mittelblau (Logo 2. Klötzchen von links)
%	\definecolor{79gruen}{rgb}{0.31,0.663,0.153} % dunkelgrün aus dem Logo
%	\definecolor{194gruen}{rgb}{0.761,0.81,0} % hellgrün
%	\definecolor{178grau}{rgb}{0.7,0.7, 0.7}
%	\definecolor{0blau}{rgb}{0 ,0.506, 0.776} % farblich gleich mit AVTcolor, dunkelblau aus Logo
%	%	
	
	%% LTT-Farben
	% Hausfarbe RWTH blau
	\definecolor{rwthblue100}{RGB}{0,84,159}
	\definecolor{rwthblue75}{RGB}{64,127,183}
	\definecolor{rwthblue50}{RGB}{142,186,229}
	\definecolor{rwthblue25}{RGB}{199,221,242}
	\definecolor{rwthblue10}{RGB}{232,241,250}
	
	% schwarz
	\definecolor{black100}{RGB}{0,0,0}
	\definecolor{black75}{RGB}{100,101,103}
	\definecolor{black50}{RGB}{156,158,159}
	\definecolor{black25}{RGB}{207,209,210}
	\definecolor{black10}{RGB}{236,237,237}
	
	% magenta
	\definecolor{magenta100}{RGB}{227,0,102}
	\definecolor{magenta75}{RGB}{233,96,136}
	\definecolor{magenta50}{RGB}{241,158,177}
	\definecolor{magenta25}{RGB}{249,210,218}
	\definecolor{magenta10}{RGB}{253,238,240}
	
	% gelb
	\definecolor{yellow100}{RGB}{255,237,0}
	\definecolor{yellow75}{RGB}{255,240,85}
	\definecolor{yellow50}{RGB}{255,245,155}
	\definecolor{yellow25}{RGB}{255,250,209}
	\definecolor{yellow10}{RGB}{255,253,238}
	
	% petrol
	\definecolor{petrol100}{RGB}{0,97,101}
	\definecolor{petrol75}{RGB}{45,127,131}
	\definecolor{petrol50}{RGB}{125,164,167}
	\definecolor{petrol25}{RGB}{191,208,209}
	\definecolor{petrol10}{RGB}{230,236,236}
	
	% türkis
	\definecolor{cyan100}{RGB}{0,152,161}
	\definecolor{cyan75}{RGB}{0,177,183}
	\definecolor{cyan50}{RGB}{137,204,207}
	\definecolor{cyan25}{RGB}{202,231,231}
	\definecolor{cyan10}{RGB}{235,246,246}
	
	% grün
	\definecolor{green100}{RGB}{87,171,39}
	\definecolor{green75}{RGB}{141,192,96}
	\definecolor{green50}{RGB}{184,214,152}
	\definecolor{green25}{RGB}{221,235,206}
	\definecolor{green10}{RGB}{242,247,236}
	
	% maigrün
	\definecolor{maygreen100}{RGB}{189,205,0}
	\definecolor{maygreen75}{RGB}{208,217,92}
	\definecolor{maygreen50}{RGB}{224,230,154}
	\definecolor{maygreen25}{RGB}{240,243,208}
	\definecolor{maygreen10}{RGB}{249,250,237}
	
	% orange
	\definecolor{orange100}{RGB}{246,168,0}
	\definecolor{orange75}{RGB}{250,190,80}
	\definecolor{orange50}{RGB}{253,212,143}
	\definecolor{orange25}{RGB}{254,234,201}
	\definecolor{orange10}{RGB}{255,247,234}
	
	% Rot
	\definecolor{red100}{RGB}{204,7,30}
	\definecolor{red75}{RGB}{216,92,65}
	\definecolor{red50}{RGB}{230,150,121}
	\definecolor{red25}{RGB}{243,205,187}
	\definecolor{red10}{RGB}{250,235,227}
	
	% bordeaux
	\definecolor{darkred100}{RGB}{161,16,53}
	\definecolor{darkred75}{RGB}{182,82,86}
	\definecolor{darkred50}{RGB}{205,139,135}
	\definecolor{darkred25}{RGB}{229,197,192}
	\definecolor{darkred10}{RGB}{245,232,229}
	
	% Violett
	\definecolor{violet100}{RGB}{97,33,88}
	\definecolor{violet75}{RGB}{131,78,117}
	\definecolor{violet50}{RGB}{168,133,158}
	\definecolor{violet25}{RGB}{210,192,205}
	\definecolor{violet10}{RGB}{237,229,234}
	
	% Lila
	\definecolor{purple100}{RGB}{122,111,172}
	\definecolor{purple75}{RGB}{155,145,193}
	\definecolor{purple50}{RGB}{188,181,215}
	\definecolor{purple25}{RGB}{222,218,235}
	\definecolor{purple10}{RGB}{242,240,247}
	
	% durch Definitionen in Tikz-Option kann der Plot in RWTH-Farben dargestellt werden.
	% zB \draw[color=rwthblue100] (0,0)-- (2,2) macht eine diagonale Linie in dunkelblau

% Chemische Formeln
\usepackage[version=4]{mhchem}	% Für die Darstellung chemischer Formeln (muss leider entweder von Hand eingebunden werden (Adminrechte nötig)  oder das mhchem.sty-file muss in dem selben Ordner wie das LaTeX-file liegen, damit LaTeX es finden kann.
\usepackage{chemfig} % Für Strukturformeln und ähnliches

%Zusätzliche Packages tima01
\usepackage{todonotes} %hiermit kann man todos einfügen ist ganz nett z.B. an Stellen wo später Bilder hin sollen
%\usepackage[decimalsymbol=comma,digitsep=period]{siunitx}   % Nutzen von Einheiten!
%\usepackage{eurosym}